\documentclass{article}
\usepackage{natbib}

% if you need to pass options to natbib, use, e.g.:
%     \PassOptionsToPackage{numbers, compress}{natbib}
% before loading neurips_2018

% ready for submission
% \usepackage{neurips_2018}

% to compile a preprint version, e.g., for submission to arXiv, add add the
% [preprint] option:
%     \usepackage[preprint]{neurips_2018}

% to compile a camera-ready version, add the [final] option, e.g.:
     \usepackage[final]{neurips_2018}

% to avoid loading the natbib package, add option nonatbib:
%     \usepackage[nonatbib]{neurips_2018}

\usepackage[utf8]{inputenc} % allow utf-8 input
\usepackage[T1]{fontenc}    % use 8-bit T1 fonts
\usepackage{hyperref}       % hyperlinks
\usepackage{url}            % simple URL typesetting
\usepackage{booktabs}       % professional-quality tables
\usepackage{amsfonts}       % blackboard math symbols
\usepackage{nicefrac}       % compact symbols for 1/2, etc.
\usepackage{microtype}      % microtypography

\title{Denoising face recognition via clustering techniques}

% The \author macro works with any number of authors. There are two commands
% used to separate the names and addresses of multiple authors: \And and \AND.
%
% Using \And between authors leaves it to LaTeX to determine where to break the
% lines. Using \AND forces a line break at that point. So, if LaTeX puts 3 of 4
% authors names on the first line, and the last on the second line, try using
% \AND instead of \And before the third author name.

\author{Trung Vu, Sean Lee, Alexandre Lamy%
  % David S.~Hippocampus\thanks{Use footnote for providing further information
  %   about author (webpage, alternative address)---\emph{not} for acknowledging
  %   funding agencies.} \\
  % Department of Computer Science\\
  % Cranberry-Lemon University\\
  % Pittsburgh, PA 15213 \\
  % \texttt{hippo@cs.cranberry-lemon.edu} \\
  % examples of more authors
  % \And
  % Coauthor \\
  % Affiliation \\
  % Address \\
  % \texttt{email} \\
  % \AND
  % Coauthor \\
  % Affiliation \\
  % Address \\
  % \texttt{email} \\
  % \And
  % Coauthor \\
  % Affiliation \\
  % Address \\
  % \texttt{email} \\
  % \And
  % Coauthor \\
  % Affiliation \\
  % Address \\
  % \texttt{email} \\
}

\begin{document}
% \nipsfinalcopy is no longer used

\maketitle

\begin{abstract}
  In this paper we address the problem of detecting and recognizing specific objects (such as the faces of a specific group of people)
  in real life videos. By using unsupervised clustering techniques, we are able to exploit the inter-frame relationships in the 
  video to significantly improve the accuracy of basic object detection and recognition algorithms. This yields results which are
  also much more robust to frame switches, lengthy occlusions, and variable numbers of targeted objects leaving and reentering the frames 
  (all common occurrences in real life videos) than standard tracking techniques.
\end{abstract}

\section{Introduction}
% setting
% problem statement
% overview of issues and our solution

% Alex


One of the most interesting and difficult problems in computer vision is that of recognizing and then tracking an
entity of interest over time (i.e. in videos). This can be useful for surveillance of suspect individuals,
analysis of video data (music videos, concerts, sports games, etc.), or in building robots that interact with their 
environment. 

While tracking is a topic that has been extensively studied (see \cite{benchmarksurvey}), it is usually
done so in isolation from recognition. To track the object of interest, the leading tracking algorithms mainly rely on 
properties of videos, notably that tracking target(s)
will move by small amounts at a time. This allows the algorithms to detect/infer the object(s) positions from the previous ones by
focusing on the region around the previously known position. While, these methods can work remarkably well on a clean video, such techniques have 
little hope when faced with long occlusions or frequent ``frame switches'' as these will cause a violation of the assumptions that the algorithms are based on.

Alternatively, object detection techniques could be applied frame by frame. This would result in a much more robust result since the algorithms
make no assumptions concerning the relations between different frames. Hence, occlusions and frame switches pose no issue whatsoever. Unfortunately, even 
the best of these methods result in relatively noisy or inaccurate results when compared with the tracking algorithms. This can be attributed to the fact that the detection
algorithms make no use of the wealth of information provided by the inter frame relationships (notably that most objects will usually not move by a large distance).

In this paper we propose a novel post-processing method, based on unsupervised clustering techniques, to ``denoise'' results obtained 
from object detection algorithms by exploiting inter-frame relationships. The result is a technique that is more robust to occlusions and frame switches than the standard 
tracking techniques and more accurate (less noisy) than the results obtained by naively applying object detection algorithms. We also have the added benefit of being able to easily
do object recognition at the same time as detection. Our method will not only reduce the noise in the detection error (bad bounding boxes) but also in the classification error (bad labels).

\section{Standard tracking methods and issues}
% tracking really isn't robust for complicated videos (varying number of people, frame switches, etc.)

Our problem, that of detecting objects throughout a video, or variations of it are often solved via tracking techniques. As mentioned above, the core idea is to exploit
inter-frame relations, notably the assumption that most objects will not move by much between frames, to track or follow the various target objects. We quickly summarize
the main tracking algorithms used in practice and show that they rely on core assumptions which are violated 
when faced with real life videos which contain frame switches, long occlusions, and various target objects coming in and out 
of the frames in variable number. In the presence of these realities, we show that these tracking algorithms are unusable.

As mentioned in \cite{benchmarksurvey}, the most commonly used tracking algorithms are OLB \cite{OLB},IVT \cite{IVT}, MIL \cite{miltrack}, L1 \cite{L1}, and TLD \cite{TLD}.
We give a quick summary of each and outline their main issue in our context.


\cite{miltrack} % trung
\cite{OLB} % alex
\cite{IVT} % alex
\cite{L1} % alex
\cite{TLD} % trung


\section{Standard object detection and recognition methods}
% Face net or MTCNN
% Nice because frame by frame, hence doesnt assume relationships, hence more robust
% BUT noisy and no coherent entity

% Trung (get some papers too)

\section{Our method: using clustering to exploit video structure in order to denoise detection and recognition methods}

% Alex

\section{Experiments and Results}

\subsubsection*{Acknowledgments}

\section*{References}

\bibliographystyle{unsrt}
\bibliography{final_report}

\end{document}
