\documentclass{article}
\usepackage{natbib}

% if you need to pass options to natbib, use, e.g.:
%     \PassOptionsToPackage{numbers, compress}{natbib}
% before loading neurips_2018

% ready for submission
% \usepackage{neurips_2018}

% to compile a preprint version, e.g., for submission to arXiv, add add the
% [preprint] option:
%     \usepackage[preprint]{neurips_2018}

% to compile a camera-ready version, add the [final] option, e.g.:
     \usepackage[final]{neurips_2018}

% to avoid loading the natbib package, add option nonatbib:
%     \usepackage[nonatbib]{neurips_2018}

\usepackage[utf8]{inputenc} % allow utf-8 input
\usepackage[T1]{fontenc}    % use 8-bit T1 fonts
\usepackage{hyperref}       % hyperlinks
\usepackage{url}            % simple URL typesetting
\usepackage{booktabs}       % professional-quality tables
\usepackage{amsfonts}       % blackboard math symbols
\usepackage{nicefrac}       % compact symbols for 1/2, etc.
\usepackage{microtype}      % microtypography

\title{Denoising face recognition via clustering techniques}

% The \author macro works with any number of authors. There are two commands
% used to separate the names and addresses of multiple authors: \And and \AND.
%
% Using \And between authors leaves it to LaTeX to determine where to break the
% lines. Using \AND forces a line break at that point. So, if LaTeX puts 3 of 4
% authors names on the first line, and the last on the second line, try using
% \AND instead of \And before the third author name.

\author{Trung Vu, Sean Lee, Alexandre Lamy%
  % David S.~Hippocampus\thanks{Use footnote for providing further information
  %   about author (webpage, alternative address)---\emph{not} for acknowledging
  %   funding agencies.} \\
  % Department of Computer Science\\
  % Cranberry-Lemon University\\
  % Pittsburgh, PA 15213 \\
  % \texttt{hippo@cs.cranberry-lemon.edu} \\
  % examples of more authors
  % \And
  % Coauthor \\
  % Affiliation \\
  % Address \\
  % \texttt{email} \\
  % \AND
  % Coauthor \\
  % Affiliation \\
  % Address \\
  % \texttt{email} \\
  % \And
  % Coauthor \\
  % Affiliation \\
  % Address \\
  % \texttt{email} \\
  % \And
  % Coauthor \\
  % Affiliation \\
  % Address \\
  % \texttt{email} \\
}

\begin{document}
% \nipsfinalcopy is no longer used

\maketitle

\begin{abstract}
  TODO abstract
\end{abstract}

\section{Introduction}
% setting
% problem statement
% overview of issues and our solution

% Alex

\section{Standard tracking methods and issues}
% tracking really isn't robust for complicated videos (varying number of people, frame switches, etc.)

As mentioned in survey \cite{benchmarksurvey}.


\cite{miltrack} % trung
\cite{OLB} % alex
\cite{IVT} % alex
\cite{L1} % alex
\cite{TLD} % trung

\newpage

\section{Standard object detection and recognition methods}
% Face net or MTCNN
% Nice because frame by frame, hence doesnt assume relationships, hence more robust
% BUT noisy and no coherent entity

% Trung (get some papers too)

\section{Our method: using clustering to exploit video structure in order to denoise detection and recognition methods}

% Alex

\section{Experiments and Results}

\subsubsection*{Acknowledgments}

\section*{References}

\bibliographystyle{unsrt}
\bibliography{final_report}

\end{document}
